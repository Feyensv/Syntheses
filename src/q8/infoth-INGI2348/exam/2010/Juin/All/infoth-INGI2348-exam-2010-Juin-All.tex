\documentclass[en]{../../../../../../eplexam}

\usepackage[SIunits]{../../../../../../eplunits}

\hypertitle{Information theory and coding}{8}{INGI}{2348}{2010}{Juin}
{John de Wasseige \and Beno\^it Legat}
{Benoît Macq, Jérôme Louveaux Jérôme and Pereira Olivier}

\paragraph{Discussion link}
\url{http://www.forum-epl.be/viewtopic.php?t=13086}

\section{BM-1}
\begin{solution}
  \begin{enumerate}
    \item
      We have
      \[
        \begin{array}{lccc}
          & 0 & 1 & 2\\
          \Pr[X = \cdot] & 0.3 & 0.4 & 0.3\\
          \Pr[Y = \cdot] & 0.3 & 0.4 & 0.3
        \end{array}
      \]
      so
      \begin{align*}
        H(X) & = H(Y)\\
             & = \log_2(10) - 6/10 \log_2(3) - 2/5 \log_2(4)\\
             & = 1/5 + \log_2(5) - 3/5 \log_2(3) \approx 1.57.
      \end{align*}
    \item
      We have
      \[
        \begin{array}{lccc}
          \Pr(X|Y) & 0 & 1 & 2\\
          0 & 2/3 & 1/4 & 0\\
          1 & 1/3 & 1/2 & 1/3\\
          2 & 0   & 1/4 & 2/3\\
        \end{array}
      \]
      so
      \begin{align*}
        H(X|Y)
        & = 6/10 (\log_2(3) - 2/3\log_2(2)) + 2/5 (\log_2(4) - 1/2 \log_2(2))\\
        & = 1/5 + 3/5 \log_2(3) \approx 1.15.
      \end{align*}
      and since since the matrix $P(X,Y)$ is symmetric,
      $H(Y|X) = H(X|Y)$.
    \item
      \[ H(X,Y) = H(Y) + H(X|Y) = 1/5 + \log_2(5) - 3/5 \log_2(3) + 1/5 + 3/5 \log_2(3) = 2/5 + \log_2(5). \]
    \item
      \[ I(X;Y) = H(X) - H(H|Y) = 1/5 + \log_2(5) - 3/5 \log_2(3) - 1/5 - 3/5 \log_2(3) = \log_2(5) - 6/5 \log_2(3) \approx 0.42. \]
    \item We can design an Huffman code.
      \[
        \begin{array}{ll}
          (0,0) & 00\\
          (1,1) & 010\\
          (2,2) & 011\\
          (1,0) & 100\\
          (0,1) & 101\\
          (2,1) & 110\\
          (1,2) & 111\\
        \end{array}
      \]
  \end{enumerate}
\end{solution}

\section{BM-2}
\begin{solution}
  Voir cours.
\end{solution}

\section{JL-1}
\begin{solution}
  \begin{enumerate}
    \item We have $A = \{0,1\}$ and $B = \{0,1,\psi\}$.
      The transition matrix is
      \[
        \mathcal{P} =
        \begin{pmatrix}
          0.8 & 0   & 0.2\\
          0   & 0.8 & 0.2
        \end{pmatrix}
      \]
      which is symmetric so the capacity is reached for
      an uniform input $\Pr[X = 0] = \Pr[X = 1] = 1/2$ so
      $\Pr[Y = 0] = \Pr[Y = 1] = 2/5$ and $\Pr[Y = \psi] = 1/5$.
      We have $H(Y) = \log_2(5) - 4/5$
      and $H(Y|X) = \log_2(5) - 8/5$ so
      $C = I(X;Y) = H(Y) - H(Y|X) = 4/5$.
    \item Yes, $R = \tau_C/\tau_s$ is not fixed
    (the time between channel letters $\tau_C$ is not fixed yet).
    \item Yes, as $K$ remains to be chosen, $N$ too and there is thus no limit on the code length.
    We have $R = 3/4$ and
      \[ \frac{C}{H(U)} = \frac{4}{5} > \frac{3}{4}. \]
    \item No, we need to be able to increase $N$ and $K$ (keeping $R$ constant).
    \item Remember that
      \[ \frac{H(U)}{\tau_s} = \frac{1}{\tau_s^*}. \]
      Here $H(U) = \si{1}{b/symbol}$ and
      $\tau_s = \tau_s^* = \si{1}{s/symbol}$.

      Since $\tau_c = 10/\si{9}{s\per symbol}$\footnote{he says 9 bits per 10 seconds but I think he means 9 symbols},
      we have
      \[ R = \frac{\tau_c}{\tau_s} = \frac{10}{9}. \]
      Since
      \[ \frac{C}{H(U)} = \frac{4}{5} < \frac{10}{9}. \]
      we have
      \[ \mathcal{H}(\bar{\pi}) \geq H(U) - \frac{\tau_s}{\tau_c}C = 1 - \frac{C}{R} = 0.28 \]
      so $\bar{\pi} \geq 0.0485$.
  \end{enumerate}
\end{solution}

\section{JL-2}
\nosolution

\section{OP-1}
\begin{solution}
  \begin{enumerate}
    \item Recall that the multiplicative inverse of $x$ modulo $p$
    is the integer $y$ such that $x y \equiv 1 \mod p$.
    To reconstruct $s$, $P_1$ and $P_2$ compute $\alpha = b_2 - b_1 \mod p$.
    If $\alpha = 0$, then $s=0$ because $a_1 \neq a_2$.
    If $\alpha \neq 0$, then $s$ is the multiplicative inverse of $(a_1 - a_2)/(b2 - b1)$.
    \item To show that $I(s; (a_1, b_1)) = 0$, we need to show that
    \[
      \Pr[s = i | (a_1, b_1) = (x, y)] 
      = \frac{\Pr[s = i \cap (a_1, b_1) = (x, y)]}{\Pr[(a_1, b_1) = (x, y)]} 
      =\Pr[s = i]
    \]
    Since $\{s = i \} = \{a_1 \cdot i + b_1 = a_2 \cdot i + b_2 \}$
    is independent (intersection between two lines)
    of how $(a_1,b_1)$ are chosen (uniformly in $\mathbb{Z}_p \times \mathbb{Z}_p$),
    we have that
    \[
      \Pr[s = i \cap (a_1, b_1) = (x, y)] = \Pr[s = i] \Pr[(a_1, b_1) = (x, y)]
    \]
    which proves what we want.
    The same reasoning holds for $(a_2,b_2)$.
    \item It is \emph{not} possible. More generally, it is not possible to have
    secret shares that are taken from a set $A$ which is smaller than the set of the key $B$.
    Suppose it is not the case. If $n-1$ shareholders decide to put their shares
    together, they obtain a number corresponding to one key of $B$. Knowing that the last
    missing share is in the set $A$ which has less elements than $B$, the secret values that
    can be taken are restricted and thus it violates the secret sharing scheme.
  \end{enumerate}
\end{solution}

\end{document}
