\section{Structures de données}
\subsection{Type de données}
Type de données $\equiv$ données + opérations $\rightarrow$ $T\equiv(D_T, OP_T)$\\
Données $D_T$: ensemble de valeurs possibles et Opérations $OP_T$ : ensemble d'opérations (procédures).\\
En général, les opérations de $OP_T$ peuvent modifier leurs paramètres qui sont passés en référence. Elles sont donc des procédures non-pures, avec effet. Elle modifie le plus souvent un ou plusieurs paramètres de type T.\\

Pour un type $T=(D_T, OP_T)$,\\ une opération $F(V_1:T_1,...V_n:T_n) : T' \in OP_T$:\\
$\bullet$ $F$ est \textbf{producteur} de $T$ ssi $T'=T$ \\
$.\quad \bullet$ \textbf{créateur} de $T$ ssi $T' =T$ et $T_1,...,T_n \neq T$\\
$\bullet$ $F$ est \textbf{observateur} de $T$ ssi $T'\neq T$, $T'$ non-vide \\
$\bullet$ $F$ est \textbf{modificateur} de $T$ ssi $T'$ vide \\
$\bullet$ $F$ est \textbf{mutateur} de $T$ ssi $T_i = T$ et $F$ modifie $V_i$, càd il modifie un paramètre de type T. (mutateur = modificateur)  \\

Un type $T$ est \textbf{immutable} s'il n'a aucun mutateur. Aucune opération ne modifie de paramètre de type $T$. C'est une propriété du type et pas seulement de son implémentation.\\
L'avantage est qu'on peut partager les valeurs car il n'y a plus de problème d'aliasing. L'inconvénient est qu'il y a plus de création et de destruction d'objets.\\

$\bullet$\textbf{Enregistrements}: $type T = {F_1:T_1,...,F_n:T_n}$\\
C'est un type produit: $D_T = D_{T1}*...*D_{Tn}$\\
Opérations:\\
producteur (constructeur) $newF(V_1:T_1,...,V_n:T_n):T$\\
observateur (accesseurs) $getF_i(V:T):T_i$\\
modificateurs $setF_i (V:T,V_i:T_i) modifies V$\\

$\bullet$\textbf{Unions}: $type T = F_1:T_1|...|F_n:T_n$\\
C'est un type somme (union disjointe) : $D_T = D_{T1}+...+D_{Tn}$\\
Opérations:\\
producteurs $newF(V_i:T_i):T$\\
observateurs $isF_i(V:T):bool$ ou $getF_i(V:T):T_i$
\subsection{Type de données inductifs}

$datatype T = G_1(F_{11}:T_{11},...)|...|G_n(F_{n1}:T_{n1},...)$\\
Le type inductif est algébrique, c'est une somme de produits, une union d'enregistrements. $G_1,...,G_n$ sont les générateurs du type $T$.\\
Opérations:\\
producteur (constructeur) $newG_i(V_{i1}:T_{i1},...):T$\\
observateurs $isG_i(V:T):bool$ ou $getF_{ij}(V:T):T_i$
modificateurs $setF_{ij} (V:T,V_{ij}:T_{ij}) modifies V$\\

$\bullet$ \textbf{Filtrage}: $match \quad E\{case M_1 \Rightarrow S_1 ... case M_k \Rightarrow S_k\}$\\
$M_1,...$ sont des motifs constitués de générateurs et de variables. Les variables de $M_1,...$ apparaissent dans $S_1, ...$\\
Le filtrage primitif pour un $match E$ est un filtrage avec tous les générateurs du type de $E$. Le filtrage général peut se réduire en filtrage primitif.\\
La preuve se fait comme un if-then-else mais avec autant de chemins simples que de générateurs du type de E.\\

Le \textbf{domaine} d'un type inductif est toutes les valeurs qu'on peut construire à partir des générateurs de ce type. Il faut au moins un créateur sinon $D_T = \varnothing$. \\

Soit la \textbf{relation} $<_T$ définie sur $D_T$ telle que $d<_T d'$ ssi d est un sous-terme de d'. Pour tout type inductif T, $<_T$ est bien-fondée sur $D_T$.\\
$d\prec_T d'$ ssi d est un sous-terme direct de d'. Pour tout type inductif T, $\prec_T$ est bien-fondée sur $D_T$.\\
Pour un type inductif T,\\
l'induction bien-fondé sur $<_T$ $\equiv$ induction structurelle sur T.\\
Si si P[y] est vrai pour tout sous-terme y de x\\
alors P[x]est vrai\\
Alors P[x] est vrai pour tout x.\\

Pour un type inductif T, pour une \textbf{procédure récursive}$F(V:T,...):T'\{S\}$,\\
F est définie par induction structurelle sur T si pour tous les appels récursifs $F(E,...)$ dans S, E est un sous-terme de V.\\
Si F est définie par induction structurelle Alors F se termine toujours, grâce au variant V par rapport  à $<_T$.

\section{Abstraction sur les données}
Seuls certains types sont directement supportés par le langage de programmation. En général, on doit construire une représentation du type désiré sur base de types primitifs du langage. On représente un \textbf{type abstrait} A par un \textbf{type concret} T.\\
Le type abstrait A est une structure quelconque, exprimé en mathématique et libre.\\
Le type concret T est un type du langage de la programmation exprimé dans le langage de programmation.\\
On veut représenter un type abstrait $A=(D_A,OP_A)$ sur base d'un type concret $T=(D_T,OP_T)$. Il faut alors représenter les valeurs abstraites $a \in D_A$ en valeurs concrètes $d\in D_T$. D'ailleurs, on peut avoir plusieurs $d$ possibles pour le même $a$. Et ensuite il faut implémenter les opérations abstraites $op \in OP_A$ sous forme de procédures sur les valeurs concrètes qui utilisent les opérations concrètes de $OP_T$.
\subsection{Fontion d'abstraction}
$A=(D_A,OP_A)$ représenté par $T=(D_T,OP_T)$\\
Fonction d'abstraction du type A : $abs_A:D_T \rightarrow D_A$\\
$a=abs_A(d)\in D_A$ est la valeur de A représenté par $d\in D_T$. C'est le coeur de l'abstraction, la première chose à définir. $abs_A$ est partielle, $abs_A(d)$ n'est pas défini si $d$ n'est pas une représentation valide d'une valeur de $A$. $abs_A$ n'est pas injective, il peut y avoir plusieurs représentations de la même valeur a de A.\\

$\bullet$\textbf{Relation d'équivalence} du type A :\\
$\sim_A: D_T*D_T \rightarrow Bool$\\
$d\sim_A d' \equiv abs_A(d) = abs_A(d')$\\

$\bullet$\textbf{Invariant de représentation} du type A (rep-invariant de A):\\
$ok_A: D_T \rightarrow Bool$\\
$ok_A(d) \equiv abs_A(d)$ est défini, d représente une valeur valide de A.\\ 


$\bullet$\textbf{Congruence} : Une opération F est congruente par rapport à l'équivalence $\sim$ ssi des arguments équivalents donnent des résultats équivalents.\\
Pour $F(...) : d \sim d' \Rightarrow F(d,...) \sim F(d',...)$\\
Toutes les opérations de l'abstraction doivent être congruentes par rapport à $\sim$. C'est le cas si elles représentent correctement une opération abstraite, sinon c'est une violation de l'abstraction.\\
Si on a une opération abstraite $f:D_A \rightarrow D_A$\\ et une procédure $F(V:T):T$ qui représente $f$: $abs(F(d))=f(abs(d))$\\

L'abstraction de données est à la base de la programmation orientée-objets, avec un type abstrait représenté par une classe, une représentation équivalente aux variables d'instance de la classe, les opérations abstraites représentées par les méthodes de la classe et toutes ces méthodes ont un paramètre implicite, par référence, modifiable, this.\\

La représentation de A par T préserve le rep-invariant ssi\\
$\bullet$ Tous les producteurs $F(V:T,...):T\{S\}$ préservent ok: $[ok(V)]S[ok(result)]$\\
$\bullet$ Tous les mutateurs $F(V:T,...):T'\{S\}$ préservent ok: $[ok(V)]S[ok(V)]$\\
Si tous les producteurs et mutateurs de $OP_T$ préservent ok\\
et seules les opérations de $OP_T$ sont utilisées pour produire des valeurs de T.\\
Alors toutes les valeurs de T satisfont ok.\\

Pour assurer la \textbf{modularité}, il faut préserver et masquer la représentation. La représentation n'est pas modifiable en dehors de l'implémentation de l'abstraction grâce à la localité. On peut raisonner sur l'implémentation indépendamment du reste. La représentation n'est pas visible en dehors de l'implémentation de l'abstraction, grâce à la modifiabilité. On peut modifier l'implémentation indépendamment du reste.\\

La programmation orientée-objets permet la préservation de la représentation. \textbf{L'encapsulation} rassemble la structure qui représente l'abstraction et les opérations (méthodes) associés en une classe qui représente le type abstrait. Les \textbf{modificateurs de visibilité} limitent les opérations accessibles et permettent de préserver et masquer la représentation.\\
Si la représentation est accessible en dehors de l'implémentation, on dit que l'implémentation expose la représentation. C'est une erreur de conception ! Les causes possibles sont une faille de visibilité, une référence exportée ou une référence importée.\\

$\bullet$ \textbf{Effets bienveillants}: Un type abstrait mutable a forcément des opérations avec effet, des mutateurs.\\
Un type abstrait immutable peut avoir des opérations avec effet qui change la représentation sans modifier l'abstraction, càd qui modifie d en préservant abs(d) = abs(d0).\\
Un \textbf{effet bienveillant} est un effet qui n'est pas visible en dehors de l'abstraction.\\

$\bullet$ \textbf{Représentation de l'équivalence}: On veut représenter l'égalité du type abstrait par une procédure $eq(d,d')$.\\
Si on a abs sous forme logique, on peut spécifier et prouver\\
$procedure \quad eq(d:T,d':T):\\
.\quad    pre \quad ok(d) \wedge ok(d')\\
.\quad    post \quad result \Leftrightarrow abs(d) = abs\\
\{...\}$\\
Et donc $eq \equiv \sim$ est une congruence.

$\bullet$ \textbf{Représentation canonique}: La représentation de A par T est canonique si toutes les valeurs absraites ont une représentation unique.\\
Dans ce cas, $d \sim d' \Leftrightarrow abs(d) = abs(d') \Leftrightarrow d=d'$ \\et on peut définir $eq(d,d') \{return (d=d');\}$
